It's a well-known fact of classical topology that continuous maps (onto
Hausdorff spaces) are entirely determined by their behavior on 
dense subspaces of the input domain. For instance, a continuous function 
$f : \mathbb{R} \to \mathbb{R}$ is entirely determined by its values 
on rational inputs (since $\mathbb{Q}$ is dense in $\mathbb{R}$).

This fact becomes \emph{much} more useful in pointfree topology, where there are
many more dense subspaces than in classical topology. Some dense subspaces 
might not even have any points. For instance, any probability distribution
$\mu : \mathcal{R}(A)$ over a space $A$ is entirely specified by its
restriction to the subspace $\text{Ran}(\mu)$, which is the smallest subspace
of $A$ that has probability 1 under $\mu$ \cite{simpson2012}.
If $\mu$ is a non-atomic measure,
then $\text{Ran}(\mu)$ doesn't have any points. Still, $\text{Ran}(\mu)$ will
be dense if the support of $\mu$ is all of $A$. For instance, if $\mu$ is
a normal distribution, then $\text{Ran}(\mu)$ will be dense but have no points.

Using these random subspaces can help to clean up classical probability theory.
For instance, in classical probability theory, disintegrations of a probability
distribution are only unique up to probability-1 subsets. Given a probability
distribution $\mu$ over a product space $A \times B$, let 
$\nu : \mathcal{R}(A)$ be its marginal distribution over $A$. Then 
$f : A \to \mathcal{R}(B)$ is a \emph{disintegration} of $\mu$ if
$$ \mu = \int \text{map}(\lambda y. (x, y), f(x)) d\nu(x). $$
Notice that $f$ isn't necessarily unique. Two disintegrations may differ on
any null sets. What this means is that a disintegration $f$ isn't 
guaranteed to provide actually useful information on all its arguments;
sometimes its results can be completely arbitrary. For instance, if $\nu$ is a
Dirac delta distribution on some point $a : A$, then $f$ can be arbitrary
on parts of $A$ which are away from $a$, so it would be silly to try to impute
meaning from $f$'s behavior away from $a$. This dilemma has been discussed
\cite{shan2016}.

Only by switching to pointfree topology can we solve the probability-1 dilemma.
We simply restrict $\mu : \mathcal{R}(A \times B)$ to 
$\mu' : \mathcal{R}(\text{Ran}(\nu) \times B)$, since $\text{Ran}(\nu)$ has
the special property that \emph{all} of its ``non-empty" parts have non-zero measure.
Therefore, disintegrations of $\mu'$ must in fact be unique.

Still, this might be disappointing if $\text{Ran}(\nu)$ has no points, as
sometimes we might be interested in inspecting the value of the disintegration
$f$ at points of the original space $A$. Therefore, we might be interested to
know when we can continuously extend $f : \text{Ran}(\nu) \to \mathcal{R}(B)$
to a map $g : A \to \mathcal{R}(B)$, and when must such an extension \emph{itself}
be unique?

This note focuses on the second question, giving it a slightly more general
phrasing. Given a function $f : S \to B$ from a subspace $S$ of $A$, when is
uniqueness guaranteed for continuous extensions $g : A \to B$? We will not
worry about existence for now.

We will show that if $S$ is dense in $A$ and if $B$ is Hausdorff, then
continuous extensions (if they exist) must be unique. Recall that $S$ is 
\emph{dense} iff the nucleus $j : \mathcal{O}(A) \to \mathcal{O}(A)$ that presents
it has $j(\bot) = \bot$. $B$ is \emph{Hausdorff} iff the diagonal relation on $B$
is closed.

Since $B$ is Hausdorff, there is an open subspace
$$ \beta = \{ b : B \ |\ g(b) \neq g'(b) \}, $$
and accordingly, its complement, the subspace of $B$ where $g$ and $g'$ agree,
is closed and is the equalizer $\text{eq}(g, g')$.

Every closed subspace is co-classified by an open set $V$ such that its nucleus
$j$ is defined by 
$$j(U) = U \cup V.$$

Since $g$ and $g'$ agree on the dense subspace $S$, it must be that $S$
is a subspace of $\text{eq}(g, g')$, meaning that $\text{eq}(g, g')$ is also
dense, so that its nucleus $j$ satisfies $j(\bot) = \bot$. But since
$\text{eq}(g, g')$ is also closed, it has $j(\bot) = \bot \cup V$ for some $V$,
and therefore it must be that $V = \bot$, meaning that $j$ is the identity,
and that in fact $\text{eq}(g, g')$ is the entire space $A$, which in
turn yields the desired result that $g$ and $g'$ agree everywhere.

Here is a quick counterexample to show why $B$ must be Hausdorff.
Suppose that both
$A$ and $B$ are the Sierp\'inski space $\Sigma$, and that the dense subspace
$S$ of $\Sigma$ is the one which includes only the $\text{true}$ point of
$\Sigma$ (the ``open" point). Then the function $f : S \to B$ which maps
$\text{true}$ to $\text{true}$ can be extended both the identity map as well
as the constant $\text{true}$ map, which are distinct, even
though both are continuous extensions of $f$.

To apply this to disintegration, it would be interesting to know when 
$\mathcal{R}(A)$ is Hausdorff for a given space $A$. If $A$ is overt
and discrete, then certainly $\mathcal{R}(A)$ is Hausdorff. Since $A$
is discrete, this means that for any open set $U : \mathcal{O}(A)$
and any distribution $\mu : \mathcal{R}(A)$, $\mu(U)$ is a real number
(rather than just a lower real number).
Then we can define
\begin{align*}
\cdot \neq \cdot &: \mathcal{R}(A) \times \mathcal{R}(A) \to \Sigma
\\ \mu \neq \nu &\triangleq \exists a : A, \mu(\{ a \}) \neq \nu(\{a \})
\end{align*}
which will be true if and only if $\mu$ and $\nu$ are distinct,
meaning that $\mathcal{R}(A)$ is Hausdorff.

But often $\mathcal{R}(A)$ will not be Hausdorff. Taking $A = \Sigma$,
we observe that $\mathcal{R}(\Sigma)$ is homeomorphic to the closed
subspace of the non-negative lower real numbers which are no greater
than 1, which is not Hausdorff.